
\thispagestyle{empty}

\section*{Annotation}

\begin{minipage}[t]{1\columnwidth}%
Slovak University of Technology Bratislava 

FACULTY OF INFORMATICS AND INFORMATION TECHNOLOGIES

Degree Course: \myStudyProgram\\

Author: \myName

Master's Thesis: \myTitle

Supervisor: \mySupervisor

Departmental advisor: \myDepartSupervisor

\myDate\\
\end{minipage}

%\bigskip{}

Internal network, demilitarized zone (DMZ) or data pipelines have been compromised by a threat actor and the information gathered from this incident is minimal. Knowing the threat actor's agenda (entering, potentially leaving and fulfilling a goal) is more valuable, because it may lead to enforcing the system perimeter or endpoint security. By deliberately baiting access to highly monitored isolated networks, all further activities may be learned and enlighten a security engineer. This thesis, by utilizing state of the art container orchestration mechanism - Kubernetes, designs and implements a monitored isolated environment. Understanding and logging all file system changes, process executions helps to create a timeline of events constructing a possible incident. The resulting implementation is a robust proof of concept virtualized and completely automated immutable infrastructure monitored on the host machine level. Container observation mechanisms are capable, but not accurate enough to yet determine the order of file system events. Further design and implementation is required to identify the \textit{point of enter} and \textit{exit}.


\newpage{}\thispagestyle{empty}

\newpage
\thispagestyle{empty}
\mbox{}
\newpage

\thispagestyle{empty}
\section*{Anotácia}

\begin{minipage}[t]{1\columnwidth}%
Slovenská technická univerzita v Bratislave

FAKULTA INFORMATIKY A INFORMAČNÝCH TECHNOLÓGIÍ

Študijný program: \mojStudProgram\\

Autor: \mojeMeno

Diplomová práca: \mojNazov

Vedúci diplomovej práce: \mojVeduci

Pedagogický vedúci: \mojPedagogVeduci

\mojDatum\\
\end{minipage}

%\bigskip{}

Interná sieť, demilitarizovaná zóna DMZ alebo zreťazené spracovanie údajov sú kompromitované útočníkom, a získané informácie o tomto incidente sú minimálne. Vedieť útočníkovú agendu (od vstup cez vykonanie agendy až po prípadný výstup) je velmi cenné, lebo to môže viesť k vylepšeniu informačnej bezpečnosti sieťového okruhu alebo cieľových staníc. Úmyselným lákaním útočníkov na prístup k vysoko sledovaným a izolovaným sieťam, poskytuje možnosť pre bezpečnostného experta sa poučiť zo zlovoľných aktivít. Diplomová práca, využitím vyspelého orchestračného mechanizmu kontajnerov - \texttt{Kubernetes}, sa venuje návrhu a implementácii sledovania izolovaného prostredia. Porozumenie a zaznamenávanie všetkým zmenám súboroého systému a vykonávania procesov napomáha vytváraniu časovej osi udalostí spájaných do prípadného incidentu. Výsledná implementácia je rozsiahlým dôkazom predstavy ako kompletne virtualizovaná a automatizovaná nemenná infraštruktúra monitorovaná na úrovni hostiteľa. Mechanizmy sledovania kontajnerov sú funkčné, ale nie sú dostatočne presné v udávaní poradia výskytu udalostí o zmene v súborovom systéme. Preto je nevyhnutný daľší návrh a implementácia aj pre konkrétne idetifikovanie \textit{bodu vstupu} a \textit{výstupu} zo systému.


\newpage{}\thispagestyle{empty}\medskip{}


\newpage{}

\newpage
\thispagestyle{empty}
\mbox{}
\newpage

\newpage
\thispagestyle{empty}
{\center{Tu vložiť zadanie diplomovej práce}}
{\center{Potom, vložiť finálny návrh zadania diplomovej práce}}
\newpage


\thispagestyle{empty}
\mbox{}
\newpage

