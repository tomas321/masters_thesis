\addcontentsline{toc}{chapter}{Resumé}
\chapter*{Resumé}
Sledovanie zlovoľných činiteľov nástražným systémom má dať odpoveď na otázky kedy, odkiaľ, ako dlho a čo sa dialo v celom systéme. Diplomová práca sa venuje najmä týmto otázkam a zároveň návrhu vhodného systému pre nedetekovateľné sledovanie činností. Hosťujúci operačný systém je Linux Ubuntu a hostiteľ sledovaného prostredia je systém Kubernetes. Prináša pozitíva aj negatíva ale v prvom rade ide o dôkaz využitia (z angl. \textit{proof of concept}), nakoľko Kubernetes je bežne využívana platforma mnohých spoločností.

Medzi bežné techniky monitorovania alebo analýzy činnosti systému patria systémy \textit{sandbox}, \textit{honeypoty} a podvodná technológia (z angl. \textit{deception technology}). Posledne uvedený, pomerne nový prístup, zavádza aktívny špionážny mechanizmus v závislosti od toho, či zlovoľný činiteľ vytrvá a realizuje svoju agendu. Výhoda umožniť činiteľovi túlať sa prostredím nedetekovateľne vynáša na scénu doteraz nevídané možnosti a techniky. Existujú rôzne príbuzné riešenia, ktoré riešia umiestnenie nástražného systému, motodiku sledovania či detekcie nekalej aktivity. Medzi často využívané techniky patrí emulácia alebo modifikácia služieb ako dostatočne dôveryhodné napodobeniny vhodné pre vybrané použitie (napr. mapovanie skúšaných prihlasovacích údajov, detekcia vybranej sieťovej požiadavky). Vo väčšine prípadov je následne sledovaná aktivity zrušená alebo nemôže pokračovať kvôli limitáciam implementácie.

Nakoľko sa táto práca zameriava na udržanie činiteľa v cieľovom prostredí, je nevyhnutné vyvarovať sa emulácií t.j. používať plne funkčné verzie aplikácií a služieb. Najvhodnejšie rozloženie sledovacích aspektov pripadá na základe udalostí, teda sieťových, súborových a vykonávaných procesov. Z hľadisky umiestnenia sú uvádzané rôzne možnosti, napríklad priamo na živé zariadenie v Internete alebo na lokálne zariadenie prepojené cez bránu \textit{proxy}. V obidvoch prípadoch sa jedná o rovnaký bázový systém.

Návrh systému sa skladá z bázového systému, spôsobu nasadenia, prostredí v systéme Kubernetes, sledovacie techniky a nasadenie jednotlivých častí. Bázový systém sa skladá zo štyroch virtuálnych strojov vrátane jedného dedikovaného datového uzla. Zvyšné tri spolu tvoria Kubernetes klaster sladajúci sa z dvoch pracujúcich (tzv. \textit{worker}) a jedného riadiaceho (tzv. \textit{master} alebo \textit{control-plane}) uzla. Stroje sú obsluhované virtualizačnou technológiou KVM a QEMU a ich automatizované nasadenie je realizované pomocov nástrojov Vagrant a Ansible. Jednota v nastaveniach je zabezpečená prevažne zdielaným obrazom systému tzv. \textit{Vagrant box} a následným zabezpečením potrebých balíkov a konfigurácie prostredníctvom Ansible. Uzli klastra sú označované ako \textit{master-1}, \textit{worker-1}, \textit{worker-2} a \textit{data-1} pre dátový uzol. Manuálna inštalácia a konfigurácia Kubebernetes predstavuje náročný proces, preto pre tento prípad sa používa už existujúce riešenie \textit{kubespray}, ktoré zaobstrará klaster do hotového stavu.

Cieľové prostredia sú navrhnuté prevažne v rámci Kubernetes prostriedkov \textit{Deployment}, \textit{Service} a \textit{PersistentVolumeClaim}, ktoré zabezpečujú nasadenie softvérových kontajnerov (ďalej len "kontajner"), sieťové zviditeľnenie služieb kontajnera respektíve mapovanie dátového úložiska na externý systém. Jedným z prostredí je napríklad spojenie mailový server, centrálna databáza, všeobecný Linuxový server pre správu zdieľaných súborov a riadiacia aplikácia pre repozitáre \textit{GIT} spolu s registrom pre kontajnery. Takéto prostredie nemusí vyslovene obsahovať len zraniteľné ale aj plne zabezpečné aplikácie a služby. V konečnom dôsledku na bezpečnosti takéhoto izolovaného prostredia nezáleží, a oplatí sa zahrnúť aj "deravú" aplikáciu pre zvýšenie šanci na útok.

Ako bolo vyššie spomenuté sledovanie týchto prostredí je založená na skupinách udalostí. Zároveň je sledovaný každý kontajner zvlášť a to na základe seperácie, ktorá využíva natívnu štruktúru kontajnerizácie, samotnej architektúre\textit{Docker} a operačného systému. V prípade detekcie zmien na súborovom, využíva sa detekcia udalostí kernelového subsystému \textit{inotify}, ktorý využíva nástroj \textit{fswatch}prevažne. Rozlišuje rôzne operácie (napr. vytvorenie, upravenie, zmazanie, premenovanie) nad súbormi, čo prospieva k presnejšej identifikácii aktivity. Ďalej sledovanie vykonávaných programov a proceov je reazlizované nástrojom \textit{execsnoop}, ktorý zneužíva \textit{eBPF} a jeho schopnosť číhať na systémové volanie \textit{execve}. Podobne adresované riešenie, je pre monitrovanie TCP spojení nástrojom \textit{tcptracer}, vďaka ktorému sa dá presne určite kedy, s kým, ako dlho a ktorý proces bol zodpovedný. Spolu tieto nástroje zaznamenávajú udalosti pre každý kontajner, využívajúc hodnotu tzv \textit{mount namespace}, ktorá sa líši pre každý kontajner. Poskytujú tým bežpečnostný sledovací systém aktívných útočníkov a odsledovať nové praktiky a metódy pre zvýšenie ochrany reálnych/produkčných systémov.

Cieľový používateľia tohto systému sú bezpečnostných technici prípadne analytici z oblasti informatiky. Preto je navrhnutý dodatočný nástroj pre kontrolu stavu monitorovacích služieb nazývaný \textit{sneakctl\_sever}. Ide o API server, ktorý poskytuje možnosti ovládanie sledovacích nástrojov ako služieb typu \textit{systemd}. Okrem toho je možné získavať aj dodatočné informácie o vykonávaných procesoch a štatistiky ľubovoľných súborov.

Na záver treba dodať, že táto práca sa zameriavala aj na jednoduchosť replikácie celého nasadenia do nového prostredia. Tieto procesy su zabezpečená nástrojom Ansible a zastrešujú každú operácia nasadenia, okrem systému hosťujúci virtuálne stroji.

Vysledná práca prináša veľa úspešného a pozitívne, avšak aj obmedzenia. Tie zahŕňajú detekovanie aj legitímnej aktivity aplikácií čo má za následok falošné pozitívna. Na druhj strane je takýchto udalostí radovo menej ako pri sledovaní živých systémov s legitímnov používateľskou aktivitou. Menšou limitáciou sú aj chýbajúce príkazy sledované nástrojom \textit{execsnoop}, ale nakoľko ide prevažne o skriptovacie príkazy ako \texttt{for, while, exit, shift, read} a podobne, do tejto skupinu nanešťastie patria aj podstatné príkazy napr. \texttt{echo, kill} a \texttt{exec}. Napriek tomu ide o schopný detekčný systém, kde je možné sledovať agendu práve aktívného činiteľa. Poskytuje vysokú mieru flexibility pri vytváraní prostredí pre konkrétne prípady použitia. Použité a vyvíjané podprogramy a nástroje pracujú nezávisle od aktuálneho Kubernetes prostredia.

Práca má potenciál aj do budúcna spolu s pár nápadmi na vylepšenie a dopracovanie k robustnejšiemu riešeniu. Zahŕňa to prepojenie \textit{sneakctl\_server} serverov do jedného kompaktného kontrolného systému s dodatočným klientským programom pre jednoduchšiu interakciu so všetkými uzlami klastra. Ďaľšia poteciálna práca do budúcna môže predstavit dynamické a živšie prostredia s adaptačným správaním celého prostredia v prospech činiteľa a simulovanými použivateľmi aby bol v presvedčení že nejde o nástražný a falošný systém. Pre rozšírenie obzoru sa odporúča aj zahrnúť "cenné" artefakty, ktoré môžu donútiť útočníka zotrvať a dodať viac cenných informácií o jeho agende. Napriek týmto vylepšeniam je aktuálny stav systému vysoko použiteľný aj pre odborníkov v oblasti kyberbezpečnosti hlavne ako pridaná hodnota.