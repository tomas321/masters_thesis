\chapter{Introduction}
\pagenumbering{arabic}
The conventional techniques of monitoring or analyzing system activity include sandboxes, honeypots and deception technology. The latter, a rather novel approach introduces an active spying mechanism depending on the malicious actor persevering and realizing his/her agenda. The advantage of allowing the actor to roam the environment undetected brings to light, so far, unseen patterns and techniques.

Kubernetes is the hosting technology, because it provides fast deployment of custom environments for research and production. It's a forward moving technology, therefore a vital aspect in security measures and technology. Even though this thesis does not consider the security of Kubernetes as its main objective, but building on the fact that containerization will experience an increase in usage \cite{article:container:prediction}. For example in 2020 a "capture the flag" event\footnote{\url{https://github.com/NodyHub/k8s-ctf-rocks}}, regarding the security of a Kubernetes cluster, proves the concerns made against the ever-increasing "container trend". Current and upcoming Kubernetes vulnerabilities \cite{thesis:k8s:vulnerabs} are the driving factor to engage the this as bait base system.

This thesis primarily focuses on a proof of concept realization. The ability to detect and monitor unauthorized actor and the usability of the proposed environments, tools and techniques. The aim is to create a compact environment, fully deployable with a automation tool, because it works towards higher utilization and simplicity in case of migration and sharing.

In the first part, \autoref{anal} and \autoref{related} discusses the malware analysis techniques, virtualization technology and the state-of-the-art Kubernetes. Regarding security and existing solutions it outlines and describes the needed knowledge to design the bait system with proper monitoring. After thorough research the solution can be designed (\autoref{design}), including the architecture, chosen technology, monitoring techniques and the emplacement in existing environments. \autoref{implementation} follows the designed components and describes the implementation with an additional \autoref{implementation:poc} exploring a realized experiment proving its applicability.