\chapter{Conclusion \label{conclusion}}
This chapter briefly summarizes the limitations and advantages of certain aspects of the proposed and implemented solution. \autoref{conclusion:future} introduces some ideas for future work and space for improvements towards a truly proactive security technology. 

\section{Summary \label{conclusion:summary}}
The bait system is a compact solution, well documented and automatically deployable. Although there are few limitations, such as emplacement of this setup under a cloud provider, where access to the lower layer (hosting OS) is improbable. Therefore isage of this solution is targeted pm custom environments posing as a legitimate cluster with transparent monitoring techniques.

Technically, the solution is mostly build on open-source tools and libraries coming also with additional drawbacks. The fswatch utility is bringing a large resource overhead by consuming higher amount of CPU, which is devastating for limited development environment. A favored solution is an increased latency - sleep interval until new events are checked. Regarding process monitoring with execsnoop, not all commands are being recorded, concretely the builtin scripting commands (displayed by \texttt{help}) e.g. \texttt{echo}, \texttt{pwd}, \texttt{cd}, \texttt{exit}. Fortunately, the whole image of the actor's agenda can be derived from the detectable programs/commands. The unsupported are not used for changing the system state (e.g. file and network operations), except for the \texttt{echo} command potential used for writing to files and \texttt{kill} with \texttt{exec} used for stopping processes and executing a command in a different environment respectively..

On the positive side, the usage of Ansible, Vagrant, kubespray and custom scripts makes the solution available to a higher range of potential users for its automated deployment and provisioning. Monitoring techniques are effective and deliver the metadata and indicators of compromise throughout an activity of the malicious actor. Management of the monitoring tools is made easy with the \textit{sneakctl\_server}. The Kubernetes environment is well designed to provide service publishing over the simulated public (from cluster's point of view) IP address. The overall monitoring design provides a transparent way of tapping the native concepts of the Linux operating system and remain undetected by an attacker.

\section{Future Work \label{conclusion:future}}
%- sneakctl client for combining all servers.. acts as a reverse proxy to all sneakctl server
%- user activity simulation - very far fetched and optimistic, since the idea is that any activity is considered suspicious
%- dynamic environments

Merging the sneakctl\_server servers in a single ecosystem controlled by a client tool. This would provide a seamless management on multiple nodes and a more user friendly interface, in comparison to the pure REST API.

A malicious actor roaming the system could become suspicious of the "quiet space", therefore another improvement could be the introduction of user activity simulation. Although, it is very far fetched, optimistic and beats the idea, that any activity is considered as unlawful.

On the other hand, a different approach to improving the "attacker's experience" are dynamic environments. The idea is to proactively create queried services if they do not exist yet. Such functionality could be theoretically realized in the Kubernetes ecosystem, since the creation of a container fast enough to pose as a higher latency anomaly.