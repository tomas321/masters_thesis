\chapter{Related Work \label{related}}
Honeypot, sandbox and deception technology make up the leading techniques in dynamic malware analysis. They differ in the scope of knowledge before the analysis at hand. Some know the filename, malware type, other artifacts and possibly the expected outcome (in which case the tool observes only the behavior). Other analyze the behavior of all activity - searching for anomalies and malicious indicators of compromise. The following sections briefly introduce the malware analysis, deception technology, honeypot and other existing solutions and studies relevant to the thesis topic.

\section{Malware file analysis solutions \label{related:malware-anal}}
Malware file analysis is a dynamic procedure when the filename and possibly malware type is known. In comparison to the scope of this thesis, all use similar techniques of malicious activity observation/monitoring, but differ in use case scenarios.

\subsection{Cuckoo sandbox \label{related:malware-anal:cuckoo}}
A most common sandbox environment for malware analysis by executing a given file in an isolated environment with reporting of the outcome \cite{docs:cuckoo:what}. All files affecting mainstream operating systems i.e. Windows, MacOS, Linux and Android are supported. In addition to known artifacts, cuckoo has no interfering processes, so all traces must be followed and provide insight to the behavior. Based on the official cuckoo documentation, the system produces various results (the following artifacts are copied from the documentation site \cite{docs:cuckoo:what}):

\begin{itemize}[noitemsep]
	\item
	Traces of calls performed by all processes spawned by the malware.
	\item
	Files being created, deleted and downloaded by the malware during its execution.
	\item
	Memory dumps of the malware processes.
	\item
	Network traffic trace in PCAP format.
	\item
	Screenshots taken during the execution of the malware.
	\item
	Full memory dumps of the machines.
\end{itemize}

Despite all differences with this thesis, cuckoo's architecture consists of the management software (host machine) and a number of virtual/physical machines for analysis. It's a tool for different use case, so a comparison is insignificant.

\subsection{Droidbox \label{related:malware-anal:droidbox}}
Another open source tool droidbox \cite{git:droidbox}, sadly discontinued several years ago, analyzes android applications using Android Virtual Devices (AVD) and the android emulator, which supports the Android activity monitoring. Analyzed applications are sandboxed in the AVDs and afterwards reports the following results (the following artifacts are copied from the documentation site \cite{git:droidbox}):

\begin{itemize}[noitemsep]
	\item Hashes for the analyzed package.
	\item Incoming/outgoing network data.
	\item File read and write operations.
	\item Started services and loaded classes through DexClassLoader.
	\item Information leaks via the network, file and SMS.
	\item Circumvented permissions.
	\item Cryptographic operations performed using Android API.
	\item Listing broadcast receivers.
	\item Sent SMS and phone calls.
\end{itemize}

Overall, droidbox introduces a simple way of analyzing android applications via an existing API of the emulator.

\subsection{Virustotal \label{related:malware-anal:virustotal}}
Similarly to cuckoo, virustotal utilizes both static and dynamic malware analysis. "VirusTotal's aggregated data is the output of many different antivirus engines, website scanners, file and URL analysis tools, and user contributions" \cite{docs:virustotal}. 

\subsection{Falcon sandbox \label{related:malware-anal:falcon}}
A direct equivalent to virustoal is a less famous tool called Hybrid Analysis \footnote{\url{hybrid-analysis.com}} powered by the Falcon sandbox. Again, it's similar to cuckoo, except the anti-evasion feature \cite{blog:detonation-techs}, which allows, even sandbox-aware malware, to be analyzed despite their evasion techniques.

\subsection{Evaluation \label{related:malware-anal:eval}}
Most of these are the same in concept and usually in realization too. The sandbox analysis techniques are concentrated on malware files rather than an actor's activity in an environment sneaking around a system. Nevertheless, the collected data for reports and some architectural aspects are very valuable for this research.

\section{Active analysis \label{related:active-anal}}
This section explores existing honeypot/honeynet technologies and a recently emerged concept - deception technologies. These technologies may be divided into two categories - dynamic \cite{blog:dynamic-honeypot} and static, where the environment adapt to the scenarios or remains unchanged respectively.

\subsection{Honeystat \label{related:active-anal:honeystat}}
Honeystat\cite{paper:honeystat} is a honeypot solution observing the behavior of the Blaster worm and may be used to detect zero-day worm threats. Authors assume the infection may be described in a systematic way, so by knowing the worm agenda and steps they model the monitoring procedure. The observation is event-based with memory, disk and network events. Since there are no regular users in the system, the memory events are all interesting violations e.g. buffer overflows. Disk events are file system modifications and network events should always be infection over related outgoing traffic. Worms require a multi-host network to have spreading possibility, so honeystat is deployed in a multi-homed VMWare environment (64 VMs * 32 IP addresses = $2^{11}$ IPs) with minimal honeypots. The procedure when events are encountered is:

\begin{enumerate}[noitemsep]
	\item The honeystat is capturing memory and disk events.
	\item If a network event occurs, the honeypot is reset to stop further spread of the worm to other machines/honeypots.
	\item Any previous memory/disk event is updated with additional information from the network event.
	\item Resets ought to be faster in virtual environment. Host VM is not rebooted, only the virtual disk (VD) is kept in a suspended state before it's replaced with a fresh copy of a VD. The reset always completes before a TCP timeout.
	\item Other steps include an analysis node, which is out of scope of this thesis.
\end{enumerate}

This solution does not introduce any isolation techniques beside utilizing virtualization. The emulation mechanisms are exposing the virtualized environment via e.g. BIOS strings or MAC address. All features and considerations for honeystat are purely for worm infection detection, other infection types could require more observables.

\subsection{Honeypoint \label{related:active-anal:honeypoint}}
Service emulation is what Honeypoint \footnote{\url{https://www.microsolved.com/honeypoint}} utilizes to lure malicious actors and detect their agenda. Production services lie in the same environment as the robust architecture of Honeypoint, which can mimic a complex network environment for deceiving an attacker. The Microsolved CEO Brent Huston claims \cite{podcast:honeypoint} that having a honeypot is a great deception technology with almost no false positives, since it is expected that no legitimate user interacts with it. It means that any recorded activity should be considered suspicious, if the honeypot targets malicious actors scanning the Internet regardless of possible domain - randomly trying IP addresses and looking for a services ought to have malicious intent. Consists of various components \cite{docs:honeypoint} that could be replicated in the Kubernetes architecture design.

\subsection{Cybertrap \label{related:active-anal:cybertrap}}
A (commercial) solution Cybertrap \cite{site:cybertrap} operates as a deception technology luring attackers away from production systems. Looking apart from that services in Honeystat are emulated, Cybertrap's deployed services cannot be distinguished by the attacker. Once the malicious actor gets inside such network, all movements are tracked. In addition, the Cybertrap's network is inaccessible by regular users, so any activity within the simulated environment is consider malicious - minimal to no false positives. Cybertrap is close to the idea of the goal of this thesis - sandboxed honeynet.

\subsection{A distributed platform of high interaction honeypots and experimental results \label{related:active-anal:hih-study}}
A case study \cite{study:hih} serving as a proof of concept in live Internet traffic observes malicious actors' trends and agenda. As a monitoring technique they patched the kernel's \texttt{tty} and \texttt{exec} modules to intercept the keystrokes and system calls respectively. The architecture is 4 machines anywhere in the world working as relays to the authors' local setup of virtual honeypots. The traffic incoming to the public interface of the relay is routed to a GRE tunnel connected to the local VM.

In a SSH scenario they created a new syscall and modified the SSH server to use it in order to intercept the login credentials. Logged data is periodically copied from the VM disk to the host disk (such extractions should be undetected by the malicious actors). All login data is stored to the database of this structure:

\begin{itemize}[noitemsep]
	\item Data from each ssh login attempt.
	\item Data from each successful ssh connection - tty buffer content and tty name.
	\item Data of programs executed with parameters and the terminal in which it ran.
	\item Session data grouping ssh connections.
\end{itemize}

\subsubsection*{Experiment \label{related:active-anal:hih-study:experiment}}
In the period of 30 days, they monitored what are the most common login-password pairs when no accounts exist on the target system. It resulted in low number of high frequency pairs. Subsequently, for half a year they monitored the time it took a threat actor from a successful login to running command under the same account. The results showed some attackers managed to escalate privileges to the root account and in some cases even changing the passwords of the other accounts on the system. In addition to executed commands, they observed the attackers were attempting to download programs from websites, which were hosted in the same country as the attacker's IP address. Furthermore, below are listed general trends of successfully logged in attackers:
\begin{itemize}[noitemsep]
	\item Checked who else is active on the system.
	\item System reconnaissance - OS name and version, processor characteristics, etc.
	\item Changed the password of the current user.
	\item Install an IP scan program and scans the IP range to proceed for
	potential lateral movement.
	\item Internet Relay Chat client setup for receiving instructions.
	\item Privilege escalation attempt.
\end{itemize}

In comparison, general trends of attacker behaviors with root privileges:
	
\begin{itemize}[noitemsep]
	\item Change the root password.
	\item Setup backdoor - open a network port future logins.
	\item Checkout information about legitimate users of the computer via custom installed software.
	\item One attacker replaced the ssh client binary.
\end{itemize}

\subsection{SIPHON \label{related:active-anal:siphon}}
A case study \cite{study:siphon} on Scalable High-Interaction Physical Honeypots (SIPHON), similar to the study before, serving as a proof of concept in live Internet traffic observing the IOT related malicious intents. Leveraging Shodan to appear visible and legitimate in the eyes of malicious actors, the honeypots where based on real devices. The architecture is divided into physical IOT devices, wormholes exposed to the Internet forwarding to the IOT devices via the proxy forwarder. Technically are devices separated using VLANs 802.1Q and the wormhole to forwarder connection are via reverse ssh tunnels. As compromise countermeasures the \textit{suricata} IPS and IDS features are enabled in the local network and periodic resets of IOT devices.

They observed the influence of device listing in Shodan. The number of scans/connection attempts on the device has tripled between `one week before listing' and `one week after listing'. It proves that being visible by Shodan increases the possibility of attack reconnaissance on device at hand. Although, after two week after listing in Shodan, the connection attempts has decreased, which good piece of knowledge before implementation.

\subsection{Evaluation}
Most of the solutions proposed introduce practical and efficient techniques such as in \autoref{related:active-anal:hih-study} they modified the ssh server to use a custom system call, or monitoring in event based fashion (see \autoref{related:active-anal:honeystat}). All were similar in hiding their true intent and efficiently retrieve relevant facts towards identification of the attacks agenda.

%maybe -> https://www.alcide.io/platform/#tab-2